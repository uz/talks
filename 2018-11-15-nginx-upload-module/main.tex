\documentclass[aspectratio=169]{ctexbeamer}
\usepackage{listings}
\usetheme{Pittsburgh}

\lstdefinestyle{customc}{
  belowcaptionskip=1\baselineskip,
  breaklines=true,
  frame=L,
  xleftmargin=\parindent,
  language=C,
  showstringspaces=false,
  basicstyle=\fontsize{3.8}{3.8}\selectfont\ttfamily,
  keywordstyle=\bfseries\color{green!40!black},
  commentstyle=\itshape\color{purple!40!black},
  identifierstyle=\color{blue},
  stringstyle=\color{orange},
}

\lstdefinestyle{customasm}{
  belowcaptionskip=1\baselineskip,
  frame=L,
  xleftmargin=\parindent,
  language=[x86masm]Assembler,
  basicstyle=\footnotesize\ttfamily,
  commentstyle=\itshape\color{purple!40!black},
}

\lstset{escapechar=@,style=customc}

\AtBeginSubsection[]{
  \begin{frame}<beamer>
    \tableofcontents[currentsection, currentsubsection]
  \end{frame}
}

\title{nginx upload module介绍}
\author{容忠健}
\institute{@Bigo}
\date{\today}
\logo{\includegraphics[width=2cm]{nginx.png}}

\begin{document}

\frame[plain]{\titlepage}

\section*{概要}
\begin{frame}
  \tableofcontents[hideallsubsections]
\end{frame}

%% 10 min
\section{nginx简介}

\subsection{为什么nginx比apache性能更好?}
\begin{frame}{\subsecname}{nginx采用了与apache不同的工作模型}
  \begin{itemize}
  \item<2-> apache为每个http request创建一个线程去处理,整个处理流程是同步的
  \item<3-> nginx启用一个master进程和多个worker进程,一般有几核就启几个worker进程,整个处理流程是异步的,通过事件(网络/超时)触发,核心是一个epoll事件轮询
  \end{itemize}
\end{frame}

\subsection{nginx框架是怎样的?}

\begin{frame}{\subsecname}{nginx框架}
  \lstinputlisting[style=customc,language=C,linerange={727-778}]{nginx/src/os/unix/ngx_process_cycle.c}
\end{frame}

\begin{frame}{\subsecname}{nginx处理事件和定时器}
  \lstinputlisting[style=customc,language=C,linerange={193-195, 242-260}]{nginx/src/event/ngx_event.c}
\end{frame}

\begin{frame}{\subsecname}{nginx处理事件}
  \lstinputlisting[style=customc,language=C,linerange={783-785, 799-806, 836-842, 883-884, 895-909, 927-937}]{nginx/src/event/modules/ngx_epoll_module.c}
\end{frame}

\begin{frame}{\subsecname}{nginx接受tcp连接 (初始化)}
  \lstinputlisting[style=customc,language=C,linerange={607-609, 648-665, 773-775, 860-863, 895-899, 904-905}]{nginx/src/event/ngx_event.c}
\end{frame}

\begin{frame}{\subsecname}{nginx从接收tcp连接到开始处理请求}
  \lstinputlisting[style=customc,language=C,linerange={17-19, 48-49, 55-55, 65-65, 139-139, 214-214, 308-308, 314-315}]{nginx/src/event/ngx_event_accept.c}
  \lstinputlisting[style=customc,language=C,linerange={1696-1698, 1711-1711}]{nginx/src/http/ngx_http.c}
  \lstinputlisting[style=customc,language=C,linerange={205-207, 322-324, 351-370}]{nginx/src/http/ngx_http_request.c}
\end{frame}

\begin{frame}{\subsecname}{nginx处理请求的一系列操作}
  \lstinputlisting[style=customc,language=C,linerange={374-374, 937-937, 1217-1217, 1879-1879}]{nginx/src/http/ngx_http_request.c}
  \lstinputlisting[style=customc,language=C,linerange={801-804, 841-842}]{nginx/src/http/ngx_http_core_module.c}
\end{frame}

\subsection{nginx如何支持第三方模块?}
\begin{frame}{\subsecname}{定义阶段和接口}
\end{frame}

\subsection{编写nginx模块要注意什么?}
\begin{frame}{\subsecname}
\end{frame}

%% 30 min
\section{nginx upload module简介}
\subsection{为什么我们需要这个模块?}
\subsection{文件怎么落地?}

\end{document}

