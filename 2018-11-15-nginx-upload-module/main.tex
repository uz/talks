\documentclass[aspectratio=169]{ctexbeamer}
\usepackage{listings}
\usetheme{Pittsburgh}

\lstdefinestyle{customc}{
  belowcaptionskip=1\baselineskip,
  breaklines=true,
  frame=L,
  xleftmargin=\parindent,
  language=C,
  showstringspaces=false,
  basicstyle=\fontsize{3.8}{3.8}\selectfont\ttfamily,
  keywordstyle=\bfseries\color{green!40!black},
  commentstyle=\itshape\color{purple!40!black},
  identifierstyle=\color{blue},
  stringstyle=\color{orange},
}

\lstdefinestyle{customasm}{
  belowcaptionskip=1\baselineskip,
  frame=L,
  xleftmargin=\parindent,
  language=[x86masm]Assembler,
  basicstyle=\footnotesize\ttfamily,
  commentstyle=\itshape\color{purple!40!black},
  keywordstyle=\bfseries\color{green!40!black},
}

\lstdefinestyle{nginx}{
  belowcaptionskip=1\baselineskip,
  breaklines=true,
  frame=L,
  xleftmargin=\parindent,
  basicstyle=\fontsize{5}{5}\selectfont\ttfamily,
}

%\lstset{escapechar=@,style=customc}

\AtBeginSubsection[]{
  \begin{frame}<beamer>
    \tableofcontents[currentsection, currentsubsection]
  \end{frame}
}

\title{nginx upload module介绍}
\author{容忠健}
\institute{@Bigo}
\date{\today}
\logo{\includegraphics[width=2cm]{images/bigo.png}}

\begin{document}

\frame[plain]{\titlepage}

\section*{概要}
\begin{frame}
  \tableofcontents[hideallsubsections]
\end{frame}

%% nginx
\section{nginx简介}

\subsection{为什么nginx比apache性能更好?}
\begin{frame}{\subsecname}{nginx采用了与apache不同的工作模型}
  \begin{itemize}
  \item<2-> apache为每个http request创建一个线程去处理,整个处理流程是同步的
  \item<3-> nginx启用一个master进程和多个worker进程,一般有几核就启几个worker进程,整个处理流程是异步的,通过事件(网络/超时)触发,核心是一个epoll事件轮询
  \end{itemize}
\end{frame}

\subsection{nginx框架是怎样的?}
\begin{frame}{\subsecname}{nginx框架}
  \lstinputlisting[style=customc,language=C,linerange={727-778}]{nginx/src/os/unix/ngx_process_cycle.c}
\end{frame}

\begin{frame}{\subsecname}{nginx处理事件和定时器}
  \lstinputlisting[style=customc,language=C,linerange={193-195, 242-260}]{nginx/src/event/ngx_event.c}
\end{frame}

\begin{frame}{\subsecname}{nginx处理事件}
  \lstinputlisting[style=customc,language=C,linerange={783-785, 799-806, 836-842, 883-884, 895-909, 927-937}]{nginx/src/event/modules/ngx_epoll_module.c}
\end{frame}

\begin{frame}{\subsecname}{nginx接受tcp连接 (初始化)}
  \lstinputlisting[style=customc,language=C,linerange={607-609, 648-665, 773-775, 860-863, 895-899, 904-905}]{nginx/src/event/ngx_event.c}
\end{frame}

\begin{frame}{\subsecname}{nginx从接收tcp连接到开始处理请求}
  \lstinputlisting[style=customc,language=C,linerange={17-19, 48-49, 55-55, 65-65, 139-139, 214-214, 308-308, 314-315}]{nginx/src/event/ngx_event_accept.c}
  \lstinputlisting[style=customc,language=C,linerange={1696-1698, 1711-1711}]{nginx/src/http/ngx_http.c}
  \lstinputlisting[style=customc,language=C,linerange={205-207, 322-324, 351-370}]{nginx/src/http/ngx_http_request.c}
\end{frame}

\begin{frame}{\subsecname}{nginx处理请求的一系列操作}
  \lstinputlisting[style=customc,language=C,linerange={374-374, 937-937, 1217-1217, 1879-1879}]{nginx/src/http/ngx_http_request.c}
  \lstinputlisting[style=customc,language=C,linerange={801-804, 841-842}]{nginx/src/http/ngx_http_core_module.c}
\end{frame}

\subsection{nginx如何支持第三方模块?}
\begin{frame}{\subsecname}{通过在感兴趣的phase注册自己的handler}
  \lstinputlisting[style=customc,language=C,linerange={108-127}]{nginx/src/http/ngx_http_core_module.h}
  \begin{itemize}
  \item<2-> \lstinputlisting[style=customc,language=C,linerange={2921-2923, 2944-2944}]{nginx-upload-module-master/ngx_http_upload_module.c}
  \item<3-> \lstinputlisting[style=customc,language=C,linerange={272-274, 280-285}]{nginx/src/http/modules/ngx_http_static_module.c}
  \end{itemize}
\end{frame}

\subsection{编写nginx模块要注意什么?}
\begin{frame}{\subsecname}
  \begin{itemize}
  \item<2-> 不能有阻塞调用,会把整个worker阻塞住
  \end{itemize}
\end{frame}

%% nginx upload module
\section{nginx upload module简介}
\subsection{为什么我们需要这个模块?}
\begin{frame}{\subsecname}
  \begin{itemize}
  \item<2-> 上传的文件落地
    \begin{itemize}
    \item<3-> 节约内存
    \item<4-> 提升效率,文件落地之后就可以用sendfile系统调用,直接由内核发送数据给storage
    \item<5-> 方便后端http server对文件的调整等处理
    \end{itemize}
  \item<6-> 断点续传
    \begin{itemize}
    \item<7-> 大文件上传由于耗时长,如果不能断点续传,受网络原因影响而导致上传失败的概率提升
    \end{itemize}
  \end{itemize}
\end{frame}

\begin{frame}{\subsecname}
  \begin{table}[h]                           
    \centering
    \begin{tabular}{ r|r|r|r|r }
      \multicolumn{5}{c}{\textbf{落地 vs 内存 (qps)}} \\ \hline
           & 并发数  &   php  & c++(nginx) & c++   \\            
      100K &  1     &    14  & 22  & 152       \\     
      100K &  50    &    509 & 138 & 108          \\     
      100K &  100   &    523 & 250 & 70         \\     
      4M &  1     &    7 & 9 & 9           \\     
      4M &  50    &    26 & 17 & 12           \\     
      4M &  100   &    26 & 23 & 10          \\     
    \end{tabular}
    \caption{文件落地性能}
    \label{tab:msg1}                            
  \end{table}
\end{frame}

\subsection{配置是怎样的?}
\begin{frame}{\subsecname}
  \lstinputlisting[style=nginx]{nginx-upload.conf}
\end{frame}

\subsection{运行}

\subsection{代码结构}
\begin{frame}{\subsecname}
  \lstinputlisting[style=customc,language=C,linerange={449-461}]{nginx-upload-module-master/ngx_http_upload_module.c}
  \lstinputlisting[style=customc,language=C,linerange={2921-2923, 2944-2944}]{nginx-upload-module-master/ngx_http_upload_module.c}
  \lstinputlisting[style=customc,language=C,linerange={796-798, 811-820, 877-877, 898-905, 920-922, 927-928}]{nginx-upload-module-master/ngx_http_upload_module.c}
\end{frame}

\begin{frame}{\subsecname}
  \lstinputlisting[style=customc,basicstyle=\fontsize{4.5}{4.5}\selectfont\ttfamily,language=C,linerange={3102-3104, 3140-3143, 3173-3188, 3213-3217, 3265-3268}]{nginx-upload-module-master/ngx_http_upload_module.c}
\end{frame}

\begin{frame}
  \lstinputlisting[style=customc,language=C,linerange={3324-3326, 3341-3349, 3359-3365, 3387-3387, 3391-3395, 3406-3417, 3428-3428, 3433-3435, 3436-3448, 3449-3453, 3471-3472}]{nginx-upload-module-master/ngx_http_upload_module.c}
\end{frame}

\begin{frame}{\subsecname}
  \lstinputlisting[style=customc,language=C,linerange={3474-3476, 3480-3488, 3500-3502, 3677-3677, 3696-3697}]{nginx-upload-module-master/ngx_http_upload_module.c}
\end{frame}

\begin{frame}{\subsecname}{multipart/form-data格式}
  \lstinputlisting[style=nginx]{application-form}  
  \lstinputlisting[style=nginx]{multipart}  
\end{frame}

\begin{frame}{\subsecname}{状态机 boundary}
  \lstinputlisting[style=customc,language=C,linerange={4043-4043, 4058-4086}]{nginx-upload-module-master/ngx_http_upload_module.c}
\end{frame}
\begin{frame}{\subsecname}{状态机 headers}
  \lstinputlisting[style=customc,language=C,linerange={4087-4130}]{nginx-upload-module-master/ngx_http_upload_module.c}
\end{frame}
\begin{frame}{\subsecname}{状态机 data}
  \lstinputlisting[style=customc,language=C,linerange={4135-4180}]{nginx-upload-module-master/ngx_http_upload_module.c}
\end{frame}

\end{document}

