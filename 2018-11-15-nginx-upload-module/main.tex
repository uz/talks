\documentclass[aspectratio=169,draft]{ctexbeamer}
%%\documentclass[aspectratio=169]{ctexbeamer}
\usetheme{Pittsburgh}
%%\usetheme{Berlin}

\title{nginx upload module介绍}
\author{容忠健}
\institute{@Bigo}
\date{\today}
\logo{\includegraphics[width=2cm]{nginx.png}}

\begin{document}

\frame[plain]{\titlepage}

%% \begin{frame}{内容}
%%   \tableofcontents[hideallsubsections]
%% \end{frame}

\section*{概要}
\begin{frame}
  \tableofcontents
\end{frame}

%% 讲10分钟
\section{nginx简介}
\subsection*{原理}
\begin{frame}
  \frametitle{为什么说nginx比apache性能更好?}
  \framesubtitle{nginx采用了与apache不同的工作模型}
  \begin{itemize}
  \item<2-> apache为每个http request创建一个线程去处理,整个处理流程是同步的
  \item<3->nginx启用一个master进程和多个worker进程,一般有几核就启几个worker进程,整个处理流程是异步的,通过事件(网络/超时)触发,核心是一个epoll事件轮询
  \end{itemize}
\end{frame}
\begin{frame}
  \frametitle{nginx是如何支持第三方模块?}
  \framesubtitle{定义阶段,设定好接口}
\end{frame}
\begin{frame}
  \frametitle{编写nginx模块要注意什么?}
\end{frame}
%% 讲30分钟
\section{nginx upload module简介}
\subsection*{概念}
\subsection*{用法}
\subsection*{原理}

%% \begin{frame}[fragile]
%%   \frametitle{nginx upload module简介}{概念}
%%   \pause
%%   nginx upload module是nginx的一个第三方http模块。它的作用为以下几点:
%% \begin{itemize}
%%   \pause
%% \item 拆解出multipart/form-data方式的post请求中的文件内容,写到本地
%%   \pause
%% \item 实现了一个断点续传的协议
%% \end{itemize}
%% \end{frame}

%% \begin{frame}
%%   \frametitle{nginx upload module简介}{使用}
%% \end{frame}

%% \begin{frame}{nginx简介}
%% \end{frame}

%% \begin{frame}{nginx upload module原理}
%% \end{frame}

\begin{frame}
\end{frame}

\end{document}

